\chapter*{Závěr}
\phantomsection
\addcontentsline{toc}{chapter}{Závěr}

Cílem této semestrální práce bylo seznámení se~šiframi lehké kryptografie a~implementace vybrané šifry do jazyka VHDL. Šifry byli nejprve voleny tak, aby splňovali požadavek na šifrovaní souborů šifrou lehké kryptografie. Z~důvodu tohoto požadavku jsem zvolil blokové šifry, které jsou nejlepší proto, že je lze na rozdíl od hašovacích funkcí dešifrovat. Vybíral jsem mezi šiframi LBlock, LED, mCRYPTON, PRESENT a~SIMON. Dospěl jsem k~závěru, že nejvhodnější šifrou pro implementaci je šifra LBlock, kterou následuje šifra PRESENT. Uvedené šifry tvoří nejoptimálnější dvojici co se týká poměru náklady, výkon a~bezpečnost. Postup šifrovaní těchto dvou šifer byl detailněji rozebrán z~důvodu, že každá z~šifer využívá jinou metodu šifrovaní.

Zvolená šifra LBlock založená na Feistelově síti byla implementována ve vývojovém prostředí Vivado Design Suite. Celá šifra byla rozdělena na několik modulů starajících se o logiku transformací vstupních dat. Výstupem hlavního modulu, který je založen na sekvenčním postupném šifrovaní, jsou šifrovaná data. Ke každému modulu byli implementovány simulační testy, kdy výstupní hodnoty z~hlavního modulu odpovídaly hodnotám v dokumentaci šifry LBlock.

Semestrální práce bude v~navazující bakalářské práci rozšířena o~implementaci na vývojovou desku ZYBO\,Z7. Ze vstupů desky, jako jsou např.\,rozhraní USB 2.0, Micro SD a~jiné, budou následně načítána data. Vstupní data budou šifrovány a~uloženy zpět na tyto vstupy. 

Pokračovaní mé práce je také možné rozšířit o přidaní šifry PRESENT a jejím porovnaní se šifrou LBlock. U~tohoto rozšíření lze měřit rychlost provedení šifry na stejných vstupních datech. Dále lze provádět měření využití šifer pomocí hodnoty LUT (LookUp Table).

Závěrem bych chtěl poděkovat Ing.\,Davidu Smékalovi za cenné rady a připomínky i odbornou a metodickou pomoc, kterou mně poskytl.
