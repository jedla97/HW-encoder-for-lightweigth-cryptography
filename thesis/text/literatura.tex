% Pro sazbu seznamu literatury použijte jednu z následujících možností

%%%%%%%%%%%%%%%%%%%%%%%%%%%%%%%%%%%%%%%%%%%%%%%%%%%%%%%%%%%%%%%%%%%%%%%%%
%1) Seznam citací definovaný přímo pomocí prostředí literatura / thebibliography

\begin{thebibliography}{99}

\bibitem{Sekanina3540403779ISBN}
    \label{source:fpgaSekanina}
    SEKANINA, Lukáš, 
    2004. 
    \emph{Evolvable components: from theory to hardware implementations.} 
    Berlin: Springer-Verlag. 
    ISBN 3-540-40377-9.

\bibitem{FPGAfromMIT}
    \label{source:fpgaMIT}
    CHANDRASEKHAR, Vikram, 
    2007. 
    \emph{CAD for a 3-dimensional FPGA} [online]. 
    [cit.\,16.\,10.\,2020]. 
    Dostupné z~URL: \(<\)\url{http://dspace.mit.edu/handle/1721.1/40520}\(>\) 
    Diplomová práce. 
    Massachusetts Institute of Technology. 
    Dept. of Electrical Engineering and Computer Science.

\bibitem{electronicshub.org}
    \label{source:FPGAarchitecture}
    Introduction to FPGA | Structure, Components, Applications, 
    2020, \emph{Electronicshub.org}
    \/[online]. 
    [cit.\,16.\,10.\,2020].
    Dostupné z~URL:
    \(<\)\url{https://www.electronicshub.org/wp-content/uploads/2020/01/FPGA-Structure.jpg}\(>\)
    
\bibitem{Zynq-7000}
    \label{source:zynq-7000}
    XILINX, INC. 
    \emph{Zynq-7000 SoC Data Sheet: Overview}
    \/[online]. 
    DS190 (v1.11.1) July 2, 2018. 
    [cit.\,17.\,10.\,2020].
    Dostupné z~URL:
    \(<\)\url{https://www.xilinx.com/support/documentation/data_sheets/ds190-Zynq-7000-Overview.pdf}\(>\)
 
\bibitem{Burda9788072049257ISBN}
    \label{source:burdaUvod}
    BURDA, Karel, 
    2015. 
    \emph{Úvod do kryptografie.} 
    Brno: Akademické nakladatelství CERM. 
    ISBN 978-80-7204-925-7.
\bibitem{Burda9788021446120ISBN}
    \label{source:burdaAplikovana}
    BURDA, Karel, 
    2013. 
    \emph{Aplikovaná kryptografie.} 
    Brno: VUTIUM. 
    ISBN 978-80-214-4612-0.
    
\bibitem{Mao0130669431ISBN}
    \label{source:MaoWenbo}
    MAO, Wenbo, 
    2004. 
    \emph{Modern Cryptography: Theory and Practice.} 
    5th ed. 
    Upper Saddle River: Prentice Hall. 
    ISBN 0-13-066943-1.
    
\bibitem{Nigel9780077099879ISBN}
    \label{source:niguel}
    SMART, Nigel, 
    2003. 
    \emph{Cryptography: An Introduction.} 
    London: McGraw-Hill College. 
    ISBN 9780077099879.
    
\bibitem{HavlicekBakalarka}
    HAVLÍČEK, Jiří, 
    2013. 
    \emph{Šifrovací algoritmy lehké kryptografie} 
    [online]. 
    Brno 
    [cit.\,22.\,10.\,2020]. 
    Dostupné z~URL: \(<\)\url{https://www.vutbr.cz/studenti/zav-prace/detail/66565}\(>\)
    Bakalářská práce. 
    Vysoké učení technické v Brně, 
    Fakulta elektrotechniky a komunikačních technologií, 
    Ústav telekomunikací. 
    Vedoucí práce Ing. Vlastimil Člupek.
    
\bibitem{EncyclopediaSynchronous}
    \label{source:synchronous}
    FONTAINE, Caroline, 
    2011. 
    Synchronous Stream Cipher.
    \emph{Encyclopedia of Cryptography and Security} 
    [online]. 
    Boston, MA: Springer 
    [cit.\,22.\,10.\,2020]. 
    ISBN 978-1-4419-5905-8.
    Dostupné z~URL: doi: \(<\)\url{https://doi.org/10.1007/978-1-4419-5906-5_376}\(>\)
    
\bibitem{EncyclopediaAsynchronous}
    \label{source:asynchronous}
    FONTAINE, Caroline, 
    2011. 
    Self-Synchronizing Stream Cipher. 
    \emph{Encyclopedia of Cryptography and Security} 
    [online]. 
    Boston, MA: Springer 
    [cit.\,22.\,10.\,2020].  
    ISBN 978-1-4419-5905-8. 
    Dostupné z: doi: \(<\)\url{https://doi.org/10.1007/978-1-4419-5906-5_371}\(>\)

\bibitem{FeistelCipher}
    \label{source:FeistelCipher}
    Feistel cipher. 
    \emph{Wikipedia} 
    [online]. 
    [cit.\,3.\,11.\,2020].  
    Dostupné z~URL: \(<\)\url{https://en.wikipedia.org/wiki/Feistel_cipher#/media/File:Feistel_cipher_diagram_en.svg}\(>\)
    

\bibitem{SaldaBP}
    \label{SaldaBP}
    ŠALDA, Jakub, 
    2017. 
    \emph{Lehká kryptografie} 
    [online]. 
    Praha 
    [cit.\,6.\,11.\,2020]. 
    Dostupné z~URL: \(<\)\url{https://vskp.vse.cz/69739_lehka_kryptografie.}\(>\) 
    Bakalářská práce. 
    Vysoká škola ekonomická v Praze, Fakulta informatiky a statistiky. 
    Vedoucí práce RNDr. Radomír Palovský, CSc.
    

\bibitem{PoschmannCrypto}
    \label{source:poschmannCrypto}
    POSCHMANN, Axel, 
    2009. 
    \emph{LIGHTWEIGHT CRYPTOGRAPHY: Cryptographic Engineering for a Pervasive World} 
    [online]. 
    Bochum 
    [cit.\,3.\,1.\,2020]. 
    Dostupné z~URL: \(<\)\url{https://www.ei.ruhr-uni-bochum.de/media/crypto/veroeffentlichungen/2011/09/16/thesisp.pdf}\(>\) 
    Disertační práce. 
    Ruhr-University Bochum, 
    Germany, 
    Faculty of Electrical Engineering and Information Technology.
    
    
\bibitem{Klima}
    \label{klima}
    KLÍMA, Vlastimil, 
    2011. 
    \emph{Co je to lehká kryptografie?} 
    [online]. 
    [cit.\,3.\,11.\,2020].  
    Dostupné z~URL: \(<\)\url{https://cryptography.hyperlink.cz/2011/ST_2011_10_16_17.pdf}\(>\)

\bibitem{LightHash}
    B.T. HAMMAD, N. JAMIL, M.E. RUSLI a M.R Z`ABA, 
    2017. 
    A survey of Lightweight Cryptographic Hash Function. 
    \emph{International Journal of Scientific\(\And\)Engineering Research} 
    [online]. 
    2017(8) 
    [cit.\,5.\,11.\,2020]. 
    ISSN 2229-5518. 
    Dostupné z~URL: \(<\)\url{https://www.ijser.org/researchpaper/A-survey-of-Lightweight-Cryptographic-Hash-Function.pdf}

\bibitem{NekuzaBP}
    \label{source:NekuzaBP}
    NEKUŽA, Karel, 
    2016. 
    \emph{Odlehčená kryptografie pro embedded zařízení} 
    [online]. 
    Brno 
    [cit.\,6.\,11.\,2020].
    Dostupné z~URL: \(<\)\url{https://www.vutbr.cz/studenti/zav-prace?zp_id=93665.}\(>\)
    Bakalářská práce. 
    Vysoké učení technické v Brně, 
    Fakulta elektrotechniky a komunikačních technologií, Ústav telekomunikací. 
    Vedoucí práce Ing. Zdeněk Martinásek, Ph.D.
   
\bibitem{PRESENT}
    \label{source:PRESENT}
    BOGDANOV Andrey, L. R. KNUDSEN, G. LEANDER, Christof PAAR, Axel POSCHMANN, M.J.B. ROBSHAW, Y. SEURIN a C. VIKKELSOE, 
    2007. 
    \emph{PRESENT: An Ultra-Lightweight Block Cipher. Cryptographic Hardware and Embedded Systems - CHES 2007.}
    [online]. 
    Berlin: Springer, Berlin, Heidelberg, 
    s. 455-466 
    [cit.\,7.\,11.\,2020]. 
    ISBN 978-3-540-74735-2. 
    Dostupné~z: doi: \(<\)\url{https://doi.org/10.1007/978-3-540-74735-2_31}\(>\)
    
\bibitem{LBlock}
    \label{source:lblock}
    WENLING, Wu a Lei ZHANG, 
    2011.
    LBlock: A Lightweight Block Cipher. Applied Cryptography and Network Security 
    [online]. 
    Germany: Springer, Berlin, Heidelberg, 
    s. 327-344 
    [cit. 2020-11-08]. 
    ISBN 978-3-642-21554-4. 
    Dostupné~z: doi: \(<\)\url{https://doi.org/10.1007/978-3-642-21554-4_19}\(>\)

\end{thebibliography}

%TODO zdroj obrazek 1 https://allaboutfpga.com/fpga-architecture/

%%%%%%%%%%%%%%%%%%%%%%%%%%%%%%%%%%%%%%%%%%%%%%%%%%%%%%%%%%%%%%%%%%%%%%%%%
%%2) Seznam citací pomocí BibTeXu
%% Při použití je nutné v TeXnicCenter ve výstupním profilu aktivovat spouštění BibTeXu po překladu.
%% Definice stylu seznamu
%\bibliographystyle{unsrturl}
%% Pro českou sazbu lze použít styl czechiso.bst ze stránek
%% http://www.fit.vutbr.cz/~martinek/latex/czechiso.tar.gz
%%\bibliographystyle{czechiso}
%% Vložení souboru se seznamem citací
%\bibliography{text/literatura}
%
%% Následující příkaz je pouze pro ukázku sazby literatury při použití BibTeXu.
%% Způsobí citaci všech zdrojů v souboru odkazy.bib, i když nejsou citovány v textu.
%\nocite{*}