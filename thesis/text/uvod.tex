\chapter*{Úvod}
\phantomsection
\addcontentsline{toc}{chapter}{Úvod}
Tato semestrální práce se věnuje šifrám lehké kryptografie a~následně výběru a~implementace zvolené šifry na obvod FPGA. Práce se skládá z~pěti hlavních kapitol. První tři kapitoly se skládají z~teoretické části, kde v~posledních dvou kapitolách je popsán výběr a~implementace šifry. 

V~první kapitole jsou blíže přiblíženy obvody FPGA. První části této kapitoly je zaměřena na popis jednotlivých částí a~jejich rozložení na FPGA čipu. V~druhé části je detailněji představen a~popsán FPGA čip z~rodiny Zynq-7000 od nejznámějšího výrobce firmy Xilinx.

V~druhé kapitole jsou přiblíženy základy kryptografie, V~jejíž první části jsou definovány základní pojmy kryptografie a~v~následující části jsou popsány hašovací funkce, asymetrické a~symetrické kryptosystémy, kde symetrické jsou dále detailněji rozděleny.

V~kapitole třetí je detailněji rozebrán princip lehké kryptografie. Zde jsou obecně rozebrány kryptosystémy lehké kryptografie a~jejich známí zástupci, ze kterých jsou podrobněji popsáni dva zástupci blokových šifer.

Ve čtvrté kapitole je postup výběru algoritmu, který bude následně implementován na FPGA čip. V~první části jsou jsou rozebrány výhody a~nevýhody pro implementaci daných typů kryptosystémů pro šifrovaní zdrojových dat. V~další části jsou porovnány konkrétní algoritmy pro implementaci na FPGA čipy.

V~poslední páté kapitole je praktická implementace vybraného algoritmu. Vybraný algoritmus bude implementován na vývojářskou desku ZYBO Z7-20 a~tato implementace bude detailněji popsána. Vybraný algoritmus bude po implementaci otestován na vstupních testovacích datech a~šifrovaná data budou porovnána s~ověřenými šifrovanými daty.

%Šablona je nastavena na \emph{dvoustranný tisk}. Pokud máte nějaký závažný důvod sázet (a~zejména tisknout) jednostranně, nezapomeňte si přepnout volbu \texttt{twoside} na \texttt{oneside}!