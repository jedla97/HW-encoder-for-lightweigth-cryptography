% V tomto souboru se nastavují téměř veškeré informace, proměnné mezi studenty:
% jméno, název práce, pohlaví atd.
% Tento soubor je SDÍLENÝ mezi textem práce a prezentací k obhajobě -- netřeba něco nastavovat na dvou místech.
\usepackage{footmisc}
\usepackage{amsmath}
\usepackage{array}
\usepackage{enumitem}
\usepackage{hyperref}
\usepackage{caption}
%\usepackage[all]{hypcap} 
\usepackage[
%%% Z následujících voleb jazyka lze použít pouze jednu
  czech-english,		% originální jazyk je čeština, překlad je anglicky (výchozí)
  %english-czech,	% originální jazyk je angličtina, překlad je česky
  %slovak-english,	% originální jazyk je slovenština, překlad je anglicky
  %english-slovak,	% originální jazyk je angličtina, překlad je slovensky
%
%%% Z následujících voleb typu práce lze použít pouze jednu
  semestral,		  % semestrální práce (nesází se abstrakty, prohlášení, poděkování) (výchozí)
  %bachelor,			%	bakalářská práce
  %master,			  % diplomová práce
  %treatise,			% pojednání o dizertační práci
  %doctoral,			% dizertační práce
%
%%% Z následujících voleb zarovnání objektů lze použít pouze jednu
%  left,				  % rovnice a popisky plovoucích objektů budou zarovnány vlevo
	center,			    % rovnice a popisky plovoucích objektů budou zarovnány na střed (vychozi)
%
]{thesis}   % Balíček pro sazbu studentských prací


%%% Jméno a příjmení autora ve tvaru
%  [tituly před jménem]{Křestní}{Příjmení}[tituly za jménem]
% Pokud osoba nemá titul před/za jménem, smažte celý řetězec '[...]'
\author{Jakub}{Jedlička}

%%% Pohlaví autora/autorky
% (nepoužije se ve variantě english-czech ani english-slovak)
% Číselná hodnota: 1...žena, 0...muž
\gender{0}

%%% Jméno a příjmení vedoucího/školitele včetně titulů
%  [tituly před jménem]{Křestní}{Příjmení}[tituly za jménem]
% Pokud osoba nemá titul před/za jménem, smažte celý řetězec '[...]'
\advisor[Ing.]{David}{Smékal}

%%% Jméno a příjmení oponenta včetně titulů
%  [tituly před jménem]{Křestní}{Příjmení}[tituly za jménem]
% Pokud osoba nemá titul před/za jménem, smažte celý řetězec '[...]'
% Nastavení oponenta se uplatní pouze v prezentaci k obhajobě;
% v případě, že nechcete, aby se na titulním snímku prezentace zobrazoval oponent, pouze příkaz zakomentujte;
% u obhajoby semestrální práce se oponent nezobrazuje (jelikož neexistuje)
\opponent[doc.\ Mgr.]{Křestní}{Příjmení}[Ph.D.]

%%% Název práce
%  Parametr ve složených závorkách {} je název v originálním jazyce,
%  parametr v hranatých závorkách [] je překlad (podle toho jaký je originální jazyk)
\title[Title of Student's Thesis]{Hardwarový šifrátor pro lehkou kryptografii}

%%% Označení oboru studia
%  Parametr ve složených závorkách {} je název oboru v originálním jazyce,
%  parametr v hranatých závorkách [] je překlad
\specialization[Information Security]{Informační bezpečnost}

%%% Označení ústavu
%  Parametr ve složených závorkách {} je název ústavu v originálním jazyce,
%  parametr v hranatých závorkách [] je překlad
%\department[Department of Control and Instrumentation]{Ústav automatizace a měřicí techniky}
%\department[Department of Biomedical Engineering]{Ústav biomedicínského inženýrství}
%\department[Department of Electrical Power Engineering]{Ústav elektroenergetiky}
%\department[Department of Electrical and Electronic Technology]{Ústav elektrotechnologie}
%\department[Department of Physics]{Ústav fyziky}
%\department[Department of Foreign Languages]{Ústav jazyků}
%\department[Department of Mathematics]{Ústav matematiky}
%\department[Department of Microelectronics]{Ústav mikroelektroniky}
%\department[Department of Radio Electronics]{Ústav radioelektroniky}
%\department[Department of Theoretical and Experimental Electrical Engineering]{Ústav teoretické a experimentální elektrotechniky}
\department[Department of Telecommunications]{Ústav telekomunikací}
%\department[Department of Power Electrical and Electronic Engineering]{Ústav výkonové elektrotechniky a elektroniky}

%%% Označení fakulty
%  Parametr ve složených závorkách {} je název fakulty v originálním jazyce,
%  parametr v hranatých závorkách [] je překlad
%\faculty[Faculty of Architecture]{Fakulta architektury}
\faculty[Faculty of Electrical Engineering and~Communication]{Fakulta elektrotechniky a~komunikačních technologií}
%\faculty[Faculty of Chemistry]{Fakulta chemická}
%\faculty[Faculty of Information Technology]{Fakulta informačních technologií}
%\faculty[Faculty of Business and Management]{Fakulta podnikatelská}
%\faculty[Faculty of Civil Engineering]{Fakulta stavební}
%\faculty[Faculty of Mechanical Engineering]{Fakulta strojního inženýrství}
%\faculty[Faculty of Fine Arts]{Fakulta výtvarných umění}
%
%Nastavení logotypu (v hranatych zavorkach zkracene logo, ve slozenych plne):

\facultylogo[logo/FEKT_zkratka_barevne_PANTONE_CZ]{logo/UTKO_color_PANTONE_CZ}

%%% Rok sepsání práce
\graduateyear{2020}

%%% Datum obhajoby (uplatní se pouze v prezentaci k obhajobě)
\date{11.\,11.\,1980} 

%%% Místo obhajoby
% Na titulních stránkách bude automaticky vysázeno VELKÝMI písmeny (pokud tyto stránky sází šablona)
\city{Brno}

%%% Abstrakt
\abstract[%
Překlad abstraktu
(v~angličtině, pokud je originálním jazykem čeština či slovenština; v~češtině či slovenštině, pokud je originálním jazykem angličtina)
]{Tato semestrální práci se věnuje tématu lehké kryptografie a implementaci vybraných algoritmů pro šifrovaní na programovatelné hradlové pole.
\newline V první části jsou popsány programovatelné hradlové pole a přiblížen Xilinx řady Zynq-7000.
\newline V druhé části jsou přiblíženu pojmy a principy kryptografie, kde následně budou popsány vybrané algoritmy lehké kryptografie. Konkrétně je pozornost zaměřená na blokové a~proudové šifry a~okrajově hashovacím funkcím.
\newline V praktické části je vybrán jeden algoritmus, který je naprogramován ve VHDL pomocí simulačního nástroje Xilinx Vivado a následně implementován na FPGA čip.
}

%%% Klíčová slova
\keywrds[%
Překlad klíčových slov
(v~angličtině, pokud je originálním jazykem čeština či slovenština; v~češtině či slovenštině, pokud je originálním jazykem angličtina)
]{%
Klíčová slova v~originálním jazyce
}

%%% Poděkování
\acknowledgement{%
Rád bych poděkoval vedoucímu diplomové práce panu Ing.~XXX YYY, Ph.D.\ za odborné vedení, konzultace, trpělivost a podnětné návrhy k~práci.
}%